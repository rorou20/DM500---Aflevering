\documentclass{article}

\title{\textbf{Take-Home Eksamen DM500 Efterår2020}}

\author{Mikkel Muusmann (mimuu20), Robin Routhe (rorou20) og Phillip Edis (phedi20)}

\date{15/11/2020}

\begin{document}

\maketitle

\textbf{Eksamen 2015 februar opgave 1}

\textbf{a)}
Elementerne: 2, 4, 6, 8 er en del af mængden, da n tilhører S får man ved udregning af A's egenskab førnævnte elementer.


\vspace{5mm} %5mm vertical space

\textbf{b)}
Elementerne: 5, 8, 11, 14 er en del af mængden, da n tilhører S får man ved udregning af B's egenskab førnævnte elementer.


\vspace{5mm} %5mm vertical space

\textbf{c)}
Svaret er 8 da de kun er dette element som indgår i både A og B.

\vspace{5mm} %5mm vertical space

\textbf{d)}
Elementerne: 2, 4, 5, 6, 8, 11, 14 indgår i mængderne A og B.

\vspace{5mm} %5mm vertical space

\textbf{e)}
Elementerne: 2, 4, 6 findes hvis man trækker mængden B fra mængden A.

\vspace{5mm} %5mm vertical space

\textbf{f)}
Elementerne: 1, 3, 5, 7, 9, 10, 11, 12, 13, 14, 15 indgår ikke i mængden A.

\vspace{5mm} %5mm vertical space
	
\textbf{Eksamen 2015 februar opgave 2}

\textbf{a)}
	udsagn 1 er sandt, da man kan finde i y for hvert, hvor det gælder at $x<y$
	udsagn 2 er falsk, da der er flere y-værdier hvor $x<y$ for et hvert x
	udsagn 3 er falsk, da man ikke kan finde et y-værdi som er større end et hvert x

\vspace{5mm} %5mm vertical space

\textbf{b)}
	$\forall y \varepsilon N: \exists x \varepsilon N: x > y$

Ved negering ændres alle kvantorer og  operander og derved får man overstående ligning


\vspace{5mm} %5mm vertical space

\textbf{Eksamen 2012 januar opgave 1}

\textbf{a)}
Den er ikke bijektiv da det er en parabel, den har mere end et punkt
hvor en y værdi har 2 x værdier. 

\vspace{5mm} %5mm vertical space

\textbf{b)}
Siden den ikke er bijektiv kan den ikke inverses.

\vspace{5mm} %5mm vertical space

\textbf{c)}
Her der ligger vi de 2 forskrifter sammen og får:

$4x^2-1$

\vspace{5mm} %5mm vertical space

\textbf{d)}
Når det er at vi boller 2 funktioner tager vi den funktion til højre og sætter ind på den anden funktions x plads og ligger det sammen.

$4x^2$

\vspace{5mm} %5mm vertical space

\textbf{Eksamen januar 2009 opgave 3 + matrice}

\textbf{a)}
Da R dikterer at b = 2a,  da kan kun pare (2,4) tilhører R, da $4 = 2*2$. $R^2$ må indeholde (2,8) grundet transitivitet.

\vspace{5mm} %5mm vertical space

\textbf{b)}
Da R består af (1,2), (2,4), (3,6), (4,8), (5,10), (6,12) og (7,14). Da kan $R^2$ skrives som: (1,4), (2,8), (3,12), siden at multiplere en relation med sig selv skaber en transitiv lukning. $R^3$ (1,8). $R^4$ er tom.

\vspace{5mm} %5mm vertical space

\textbf{matrice)}
Den første række i matricen for relationen R består af A-værdierne, hvor den anden række består af B-værdierne.
\begin{tabular}{ | c | c | c | }
 \hline
 1 & 2 & 3 \\
 2 & 4 & 6 \\
 \hline
\end{tabular}

\end{document}