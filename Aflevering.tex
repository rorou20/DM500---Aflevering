\documentclass{article}

\title{\textbf{Take-Home Eksamen DM500 Efterår2020}}

\author{Mikkel Muusmann (mimuu20), Robin Routhe (rorou20) og Phillip Edis (phedi20)}

\date{15/11/2020}

\begin{document}

\maketitle


\textbf{Eksamen 2012 januar opgave 1}

\textbf{a)}

Den er ikke bijektiv da det er en parabel, den har mere end et punkt
hvor en y værdi har 2 x værdier. 

\vspace{5mm} %5mm vertical space

\textbf{b)}

Siden den ikke er bijektiv kan den ikke inverses.

\vspace{5mm} %5mm vertical space

\textbf{c)}

Her der ligger vi de 2 forskrifter sammen og får:

$4x^2-1$

\vspace{5mm} %5mm vertical space

\textbf{d)}

Når det er at vi boller 2 funktioner tager vi den funktion til højre og sætter ind på den anden funktions x plads og ligger det sammen.

$4x^2$

\vspace{5mm} %5mm vertical space

\textbf{Eksamen januar 2009 opgave 3 + matrice}

\textbf{a)}

Da R dikterer at b = 2a,  da kan kun pare (2,4) tilhører R, da $4 = 2*2$. $R^2$ må indeholde (2,8).

\vspace{5mm} %5mm vertical space

\textbf{b)}

Da R består af (1,2), (2,4), (3,6), (4,8), (5,10), (6,12) og (7,14). Da kan $R^2$ skrives som: (1,4), (2,8), (3,12). $R^3$ (1,8). $R^4$ er tom.

\vspace{5mm} %5mm vertical space

\textbf{matrice)}

\begin{tabular}{ | c | c | c | }
 \hline
 1 & 2 & 3 \\
 2 & 4 & 6 \\
 \hline
\end{tabular}

\end{document}