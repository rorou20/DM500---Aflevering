\documentclass{article}

\title{\textbf{Take-Home Eksamen DM500 Efterår2020}}

\author{Mikkel Muusmann (Mimuu20), Robin Routhe og Phillip Edis}

\date{15/11/2020}

\begin{document}

\maketitle



\textbf{Eksamen 2015 februar opgave 1}

\textbf{a)}
	2, 4, 6, 8

\textbf{b)}
	5, 8, 11, 14

\textbf{c)}
	8

\textbf{d)}
	2, 4, 5, 6, 8, 11, 14

\textbf{e)}
	2, 4, 6

\textbf{f)}
	1, 3, 5, 7, 9, 10, 11, 12, 13, 14, 15
	
\textbf{Eksamen 2015 februar opgave 2}

\textbf{a)}
	udsagn 1 er sandt, da man kan finde i y for hvert, hvor det gælder at x<y
	udsagn 2 er falsk, da der er flere y-værdier hvor x<y for et hvert x
	udsagn 3 er falsk, da man ikke kan finde et y-værdi som er større end et hvert x

\textbf{b)}
	$\forall y \varepsilon N: \exists x \varepsilon N: x > y$



\textbf{Eksamen 2012 januar opgave 1}

\textbf{a)}

Den er ikke bijektiv da det er en parabel, den har mere end et punkt
hvor en y værdi har 2 x værdier. 

\textbf{b)}

Siden den ikke er bijektiv kan den ikke inverses.

\textbf{c)}

Her der ligger vi de 2 forskrifter sammen og får:

$4x^2-1$

\textbf{d)}

Når det er at vi boller 2 funktioner tager vi den funktion til højre og sætter ind på den anden funktions x plads og ligger det sammen.

$4x^2$

\end{document}